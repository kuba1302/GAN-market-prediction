% !TEX TS-program = pdflatex
% !TEX encoding = UTF-8 Unicode

% This is a simple template for a LaTeX document using the "article" class.
% See "book", "report", "letter" for other types of document.

\documentclass[11pt]{article} % use larger type; default would be 10pt

\usepackage[utf8]{inputenc} % set input encoding (not needed with XeLaTeX)

%%% Examples of Article customizations
% These packages are optional, depending whether you want the features they provide.
% See the LaTeX Companion or other references for full information.

%%% PAGE DIMENSIONS
\usepackage{geometry} % to change the page dimensions
\geometry{a4paper} % or letterpaper (US) or a5paper or....
% \geometry{margin=2in} % for example, change the margins to 2 inches all round
% \geometry{landscape} % set up the page for landscape
%   read geometry.pdf for detailed page layout information

\usepackage{graphicx} % support the \includegraphics command and options

% \usepackage[parfill]{parskip} % Activate to begin paragraphs with an empty line rather than an indent

%%% PACKAGES
\usepackage{booktabs} % for much better looking tables
\usepackage{array} % for better arrays (eg matrices) in maths
\usepackage{paralist} % very flexible & customisable lists (eg. enumerate/itemize, etc.)
\usepackage{verbatim} % adds environment for commenting out blocks of text & for better verbatim
\usepackage{subfig} % make it possible to include more than one captioned figure/table in a single float
% These packages are all incorporated in the memoir class to one degree or another...

%%% HEADERS & FOOTERS
\usepackage{fancyhdr} % This should be set AFTER setting up the page geometry
\pagestyle{fancy} % options: empty , plain , fancy
\renewcommand{\headrulewidth}{0pt} % customise the layout...
\lhead{}\chead{}\rhead{}
\lfoot{}\cfoot{\thepage}\rfoot{}

%%% SECTION TITLE APPEARANCE
\usepackage{sectsty}
\allsectionsfont{\sffamily\mdseries\upshape} % (See the fntguide.pdf for font help)
% (This matches ConTeXt defaults)

%%% ToC (table of contents) APPEARANCE
\usepackage[nottoc,notlof,notlot]{tocbibind} % Put the bibliography in the ToC
\usepackage[titles,subfigure]{tocloft} % Alter the style of the Table of Contents
\renewcommand{\cftsecfont}{\rmfamily\mdseries\upshape}
\renewcommand{\cftsecpagefont}{\rmfamily\mdseries\upshape} % No bold!

%%% END Article customizations

%%% The "real" document content comes below...

\title{Brief Article}
\author{The Author}
%\date{} % Activate to display a given date or no date (if empty),
         % otherwise the current date is printed 

\begin{document}
\maketitle

\section{First section}

\begin{tabular}
{lllll}
\toprule
{} &              EA &                 TTWO &              UBSFY &                 ATVI \\
\midrule
0  &              EA &             Take Two &            Ubisoft &             Blizzard \\
1  &            Fifa &               NBA 2K &    Assasin's Creed &            Starcraft \\
2  &        The Sims &           Battleborn &                 AC &             Warcraft \\
3  &  Need for Speed &             BioShock &            Far Cry &            Overwatch \\
4  &             NFL &          Borderlands &         Watch Dogs &               Diablo \\
5  &            Apex &               Evolve &  Rainbow Six Siege &    World of Warcraft \\
6  &     Battlefield &                Mafia &          Wildlands &          Hearthstone \\
7  &       Bejeweled &         Civilization &          For Honor &  Heroes of the Storm \\
8  &     Battlefront &         The Darkness &       Tom Clancy's &                  \\
9  &             NBA &                 XCOM &       The Division &                  \\
10 &      Dragon Age &                  WWE &                &                  \\
11 &       Titanfall &                  GTA &                &                  \\
12 &      Dead Space &     Grand Theft Auto &                &                  \\
13 &             &            Max Payne &                &                  \\
14 &             &  Red Dead Redemption &                &                  \\
15 &             &                  RRD &                &                  \\
\bottomrule
\end{tabular}


\subsection{A subsection}

In our analysis we decided to analyze the algorithm on 4 large companies from the gaming industry. This sector was chosen due to the fact that the study takes into account the influence of sentiment analysis - one of our hypotheses is that the opinions of Reddit users have a particularly significant impact on valuations of a given company in the gaming industry. As many as 4 companies were chosen for the analysis - Electronic Arts, Ubisoft, Take-Two Interactive Software, Activision Blizzard, called in the later part of the paper by their stock ticker: EA, UBSFY, TTWO, ATVI.  The selection of more than one company was influenced by making sure that the algorithm was reproducible and universal. Furthermore, attempts were made to use the model learned on one company's data on another company's test data. These companies are among the leaders in the growth industry. The choice of only U.S. companies was due to a possible language barrier when analyzing the sentiment of companies that produce games primarily for the Asian market. Due to the use of one-day intervals during replacement, limitations of Reddit's API (Application Programming Interface), and limitations of computing power, the analysis was conducted over a nearly three-year period: from 01/01/2019 to 31/10/2021. The data comes from the US NASDAQ stock market and was retrieved using the yahoo finance library for the python language. 
\end{document}
